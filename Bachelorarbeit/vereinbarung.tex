\documentclass[a4paper, 11pt]{article}
\usepackage{a4wide}
\usepackage{ifthen}

\usepackage{ngerman}
\usepackage[latin1]{inputenc}
\usepackage[T1]{fontenc}
\usepackage{ae,aecompl}

\topmargin 0cm \textheight 23cm \parindent0cm

% ---------------------------------------------
% HIER KONSTANTEN DEFINIEREN!
% ---------------------------------------------

\newcommand{\Kandidat}{Mattes Drechsel}
%\newcommand{\KandGeschlecht}{w} %weiblich
\newcommand{\KandGeschlecht}{m} %maennlich

\newcommand{\KandStrasse}{Eichenstr. 84C}
\newcommand{\KandPlz}{26131}
\newcommand{\KandOrt}{Oldenburg}
\newcommand{\KandEmail}{mattes.drechsel@uol.de}
\newcommand{\KandTel}{0441/77030920}
\newcommand{\KandMatrNr}{5403537}
%\newcommand{\Studiengang}{Informatik}
\newcommand{\Studiengang}{Wirtschaftsinformatik}

\newcommand{\Titel}{Konzeption und Umsetzung eines Jira Plug-ins f"ur die integrierte Nutzer- und Projektverwaltung im Softwareprojekt und in Projektgruppen} % Titel der Arbeit
\newcommand{\StartDatum}{01.05.2021}
\newcommand{\Semester}{SoSe 2021} % Startet die Arbeit im "SoSe JJJJ" oder im "WiSe JJJJ/JJJJ+1"?
\newcommand{\KandFachsemester}{6} % Fachsemester zum Start der Arbeit
\newcommand{\EndDatum}{31.08.2021} % Achtung +4 bzw 6 Monate -1 Tag!

\newcommand{\Betreuer}{Dr. Marco Grawunder} %inkl. Titel
\newcommand{\Erstgutacher}{Dr. Marco Grawunder} %inkl. Titel
\newcommand{\Zweitgutachter}{Prof. Dr. J"urgen Sauer} %inkl. Titel

% Zur Steuerung der Unterschriftenbalken
\newcommand{\BetreuerGleichErstgutachter}{Ja}
\newcommand{\BetreuerGleichZweitgutachter}{Nein}

% Arbeitstyp
% entsprechend auskommentieren
%\newcommand{\Arbeitstyp}{Masterarbeit}
\newcommand{\Arbeitstyp}{Bachelorarbeit}

\begin{document}
\selectlanguage{german}

% -----------------------------------------------------------------------------
%               Titel
% -----------------------------------------------------------------------------


Universit"at Oldenburg \hfill \today

Department f. Informatik, Abt. Informationssysteme\newline
\emph{\Erstgutacher{}}\newline
\emph{\Zweitgutachter{}}\newline

\begin{center}

  \large{\bf Vereinbarung zur \Arbeitstyp{} von \Kandidat{}}

  \vspace*{0.5cm}

 \large{\bf "`\Titel{}"'}

\end{center}

\setlength{\parskip}{1.5ex plus0.5ex minus 0.5ex}

% -----------------------------------------------------------------------------
%               Text
% -----------------------------------------------------------------------------

\section{Problemstellung}

Im Studienverlaufsplan der meisten Studierenden der Informatik bzw. der Wirtschaftsinformatik an der Universit"at Oldenburg stehen verpflichtend die Module \glqq Softwareprojekt\grqq{} f"ur Studierende des Fach-Bachelors und \glqq Projektgruppe\grqq{} f"ur Studierende des Fach-Masters. Beide Module zielen darauf ab, Studierenden die M"oglichkeit zu bieten eigenst"andig gro"se Softwareprojekte nach softwaretechnischen Kriterien zu planen, zu implementieren und zu dokumentieren. F"ur die Organisation der Projektgruppen in diesen Modulen stellt die Universit"at den Studierenden einige Tools aus der Atlassian-Suite zur Verf"ugung:

\begin{itemize} \itemsep-0.1cm
\item \textbf{Jira}\\ Jira ist ein Projektmanagement-Tool, welches das Dokumentieren und Organisieren von Aufgaben erm"oglicht. Im Kontext der Module wird es genutzt, um Anforderungen an das Produkt in Form von Userstories festzuhalten und die Arbeitszeit an den einzelnen Aufgaben zu dokumentieren. Dabei hat hat jedes Projektteam ein eigenens Projekt in Jira, also einen privaten, von anderen Teams meist nicht einsehbaren Bereich \cite{jiraDoc}.
\item \textbf{Confluence}\\ Confluence ist ein Tool, um projektbezogene Wikis zu erstellen. Dabei hat jedes Projektteam einen eigenen sogenannten Space in Confluence, der meist nicht von anderen Teams einsehbar ist \cite{confluenceDoc}.
\item \textbf{Bitbucket}\\ Bitbucket ist das Atlassian-eigene Versionierungstool, welches auf Git basiert \cite{chacon2014pro}. Dabei beinhaltet Bitbucket alle g"angigen Funktionen von Git und stellt dar"uber hinaus noch eine nahtlose Integration zu Jira bereit. Dabei k"onnen aus Jira heraus Branches zu einer entsprechenden Aufgabe erstellt werden. Zum Mergen der Branches bietet Bitbucket au"serdem die M"oglichkeit sogenannte \glqq Pull Requests \grqq{} zu erstellen. Dabei tr"agt der Autor des Codes eines Branches andere Nutzer als \glqq Reviewer \grqq{} ein, welche den Code sichten und testen und schlussendlich ablehnen oder aktzeptieren. Endg"ultig akzeptierter Code kann anschlie"send gemerged werden \cite{bitbucketDoc}.
\item \textbf{Bamboo}\\ Bamboo ist Atlassians Build-Server Tool f"ur Continuous Integration und Deployment. Dabei testet Bamboo anhand von definierten Workflows den auf den Branches in Bitbucket liegenden Code, womit vermieden werden kann, fehlerhaften Code in Produktbranches zu integrieren \cite{bambooDoc}.
\end{itemize}

Im Zuge der Arbeit wird ein Plug-in f"ur das Projektmanagement-Tool Jira entwickelt. Das Plug-in soll es erm"oglichen Projekte schnell aufzusetzen, indem beispielsweise aus Stud.IP exportierte .csv-Dateien importiert werden, anhand welcher automatisch Nutzer, Gruppen und Projekte in Jira angelegt werden. Dar"uber hinaus soll es m"oglich sein, anzugeben welche zus"atzlichen Tools aus der Atlassian-Suite (Confluence, Bitbucket, Bamboo) f"ur das entsprechende Jira-Projekt eingebunden werden sollen. Anschlie"send werden Nutzer und Gruppen in Jira angelegt und von dort aus die anderen Systeme synchronisiert. Der bisherige Stand des Systems verlangt das manuelle Anlegen der meisten dieser Einstellungen.\\
Die genauen Anforderungen und Details des Produktes werden als Teil der Bachelorarbeit erhoben und dokumentiert.\\
Zur Zeit gibt es kein ver"offentlichtes Plug-in, welches genau diesen Nutzen bringt. Die Herausforderungen beim Entwickeln dieses Plug-ins werden einerseits darin liegen das Plug-in gem"a"s der Design-Richtlinien von Atlassian zu entwickeln und andererseits die geeigneten Schnittstellen zu den anderen Tools von Atlassian zu bedienen.\\
Das Plug-in wird in Java geschrieben. Das Erstellen eines Jira Plug-in-Skelettes ist von Atlassian ausf"uhrlich dokumentiert. Der csv-Import muss sinnvoll verabeitet werden und die Java-API in Jira entsprechend genutzt werden, die Projekte zu erstellen. Anschlie"send sollen die entsprechenden Bereiche, Repositories usw. in den anderen Tools ebenfalls "uber geeignete Schnittstellen angelegt werden.
% ----------------------------------------------------------------------------
\section{Einzelaufgaben und angestrebte Ergebnisse}

% \subsection*{Konzeptionell}

\begin{itemize} \itemsep-0.1cm
\item Anforderungsanalyse
\item Anlegen eines Plug-In Skelettes
\item Anlegen einer Test-Suite
\item Erstellung eines GUI-Prototypens mit erster Funktionalit"at einer Datei-Importfunktion
\item Erstellung/Klonen von Jira-Projekten anhand des hochgelandenen csv-Files
\item Erstellung von Confluence Spaces mit Verkn"upfung zu den entsprechenden Projekten
\item Erstellung von Repositories
\end{itemize}

% \subsection*{Softwaretechnisch}

% \begin{itemize} \itemsep-0.1cm
% \item Einarbeitung in die Werkzeuge zur Umsetzung des Entwurfs
% \item \ldots
% \item Test der implementierten Software
% \end{itemize}

% -----------------------------------------------------------------------------

\section {Vereinbarungen "uber den Ablauf}

\begin{itemize}
\item Es empfiehlt sich, alle 1-2 Wochen eine R"uckkopplung mit dem
  Betreuer durchzuf"uhren, bei der neben einer inhaltlichen Diskussion
  auch eine kritische Bestandsaufnahme zur Einhaltung des
  Meilensteinplans erfolgen sollte.
\item \ifthenelse{\equal{\Arbeitstyp}{Diplomarbeit} \or \equal{\Arbeitstyp}{Masterarbeit}}{Der Zwischenstand
  zur Halbzeit ist im Rahmen eines Gespr"aches mit Gutachtern und
  Betreuer zu pr"ufen.}{}
  Die Endergebnisse der Arbeit sind nach Abschluss der Bearbeitung in Form
  eines selbst\-st"andigen Referats%
  \ifthenelse{\equal{\Arbeitstyp}{Diplomarbeit} \or \equal{\Arbeitstyp}{Masterarbeit}}
  { im Rahmen des Oberseminars}{ im Rahmen eines Forschungsseminars}
  zu pr"asentieren.
\item Evtl. zu implementierende Programme sind nach Vorgabe des betreuenden
  Mitarbeiters in bestehende Systemumgebungen zu integrieren und zu testen.
\item Den Implementierungsarbeiten ist ein ingenieurm"a"siges
  Softwaretechnik-Konzept zu\-grunde zu legen. Insbesondere ist die Imple\-men\-tierung
  durch eine geeignete Online-Doku\-men\-ta\-tion nachzuweisen.
\item Hinweis: Die Bearbeitung
  einer \Arbeitstyp{} ist eine Vollzeit-Besch"aftigung (Sie sollten deshalb
  %mindestens
  entsprechend der eigenen Leistungsf"ahigkeit
  \ifthenelse{\equal{\Arbeitstyp}{Masterarbeit}}{40}{30}
  oder mehr
  Stunden Arbeitszeit je Woche einplanen).
\item F"ur die Arbeit sind die Layout-Vorgaben der Abteilung (Style-File) einzuhalten.
\end{itemize}

% -----------------------------------------------------------------------------

\section {Organisatorisches}

\begin{tabbing}
Dauer der Arbeit: \hspace{1.3cm} \= \StartDatum\ -- \EndDatum\\
\vspace{0.5ex}Student: \> \Kandidat\\
\> \KandStrasse\\
\> \KandPlz{} \KandOrt\\
\> E-Mail: \KandEmail{}\\
\> Tel: \KandTel{}\\
\> Matr.-Nr.: \KandMatrNr\\
\> Studiengang: \Studiengang\\
\> Das \Semester\ ist mein \KandFachsemester . Fachsemester.\\[1ex]
%\vspace{0.5ex}
Betreuer: \> \Betreuer\\[1ex]
Gutachter: \> \Erstgutacher{}, \Zweitgutachter\\
\end{tabbing}

% -----------------------------------------------------------------------------

\section {Notwendige Voraussetzungen}

% {\bf Hardware:}
% \begin{itemize}
% \item \ldots
% \end{itemize}

{\bf Software:}
\begin{itemize} \itemsep-0.5ex
\item Jira-Testsystem. U.U. auf einer VM laufend.
\end{itemize}

% -----------------------------------------------------------------------------

\section {Vorl"aufige Gliederung}

\begin{enumerate}

\item Einleitung
\item Atlassian-Suite
\item Vorhandene Schnittstellen (Rest, Java)
\item Jira-Plugin
\item Testing
\item Fazit

\end{enumerate}

% -----------------------------------------------------------------------------

\section {Meilensteinplan}

\begin{tabbing}
01.05.21 \= - \= 31.05.21 \= Anforderungsanalyse u. Plug-in Prototyp\\
\> \> \> Prototyp kann Jira-Projekte Klonen\\
\> \> \> Prototyp kann Nutzer in Jira anlegen\\
01.06.21\> - \>30.06.21\>Plug-in spricht Confluence u. Bitbucket an\\
\> \> \> Prototyp legt Confluence-Space an\\
\> \> \> Prototyp legt Repositories f"ur Bitbucket an\\
01.07.21\> - \>31.07.21\>Plug-in spricht Bamboo an u. Testkonzept\\
\> \> \> Prototyp legt Bamboo-Workflows an\\
\> \> \> Ausarbeitung eines Test-Konzeptes\\
01.08.21\> - \>31.08.21\>Bug-fixing u. Testing\\
\> \> \> \ldots\\
\end{tabbing}

% -----------------------------------------------------------------------------
%               Literaturliste
% -----------------------------------------------------------------------------
\addcontentsline{toc}{chapter}{Literatur}
\bibliography{Lit}
\bibliographystyle{alphadin}

% -----------------------------------------------------------------------------
%               Unterschriften
% -----------------------------------------------------------------------------
\begin{samepage}
\vspace*{3cm}

\begin{tabular}{ccc}
  --------------------------------------------------- &  & \ifthenelse{\equal{\BetreuerGleichZweitgutachter}{Nein}}{---------------------------------------------------}{ } \\
  \Kandidat{} &  & \ifthenelse{\equal{\BetreuerGleichZweitgutachter}{Nein}}{\Zweitgutachter{}}{ }  \\ \vspace{3cm}
   &  &   \\
  --------------------------------------------------- &  & \ifthenelse{\equal{\BetreuerGleichErstgutachter}{Nein}}{---------------------------------------------------}{ }\\
  \Betreuer{} &  & \ifthenelse{\equal{\BetreuerGleichErstgutachter}{Nein}}{\Erstgutacher{}}{ } \\
\end{tabular}
\end{samepage}


\end{document}


% LocalWords:  Rechnerproblemen Softwarelabor Department Meilensteinplans
